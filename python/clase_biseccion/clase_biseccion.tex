% Procesar con lualatex. Evita los problemas con Unicode.
\documentclass[11pt]{beamer}

\mode<presentation>
{ \usetheme{Madrid} }

\usepackage[spanish]{babel}
%\usepackage[utf8x]{inputenc}
\usepackage{default}
\newtheorem{teorema}{Teorema}
\newtheorem{observacion}{Observación}
\newtheorem{definicion}{Definición}
\newtheorem{lema}{Lema}
\usepackage{verbatim}

\newcommand{\ep}{\varepsilon}
\newcommand{\eps}[1]{{#1}_{\varepsilon}}
\def\R{\mathbb{R}}
\def\N{\mathbb{N}}
\def\Z{\mathbb{Z}}

\def \a{\alpha}
\def \b{\beta}
\newcommand{\be}{\begin{equation}}
\newcommand{\ee}{\end{equation}}


\usepackage[utf8x]{inputenc}
%\usepackage{default}
\usepackage {graphics}
\usepackage[procnames]{listings}
\usepackage{pifont} 

\definecolor{orange}{rgb}{1,0.5,0} 
\definecolor{gelb}{rgb}{1,1,0}
\definecolor{blaugrau}{rgb}{0,0.5,0.5}
\definecolor{dgreen}{rgb}{0,.8,0.2}
\definecolor{hellblau}{rgb}{0,0.8,1}
\definecolor{hellblau2}{rgb}{0,0.5,1}
\definecolor{violet}{rgb}{0.7,0,0.5}
\definecolor{lila}{rgb}{1,0,1}
\definecolor{grau}{rgb}{0.3,0.3,0.3}
\definecolor{dlila}{rgb}{0.8,0.0,0.4} 
\definecolor{braun}{rgb}{0.9,0.6,0} 
\definecolor{braunlila}{rgb}{0.6,0.3,0.2} 

\def\azul{\color{blue}}
\def\rojo{\color{red}}
\def\negro{\color{black}}
\def\verde{\color{green}}

\newcommand{\mcd}{\hbox{mcd}}

\begin{document}


\definecolor{keywords}{RGB}{255,0,90}
\definecolor{comments}{RGB}{0,0,113}
\definecolor{red}{RGB}{160,0,0}
\definecolor{green}{RGB}{0,150,0}
 
\lstset{language=Python, 
        basicstyle=\ttfamily\small, 
        keywordstyle=\color{keywords},
        commentstyle=\color{comments},
        stringstyle=\color{red},
        showstringspaces=false,
        identifierstyle=\color{green},
        procnamekeys={def,class}}
 

\title[Bisección]{El Método de Bisección (método numérico para el teorema de Bolzano) }
\author{Pablo L. De Nápoli}
\date{9 de septiembre de 2019}


\institute[DM-UBA]{Departamento de Matem\'atica \\
  Facultad de Ciencias Exactas y Naturales \\
  Universidad de Buenos Aires }

%{pdenapo@dm.uba.ar}
\maketitle

\part{El Programa para el método de Bisección}

\frame{\partpage}
\begin{frame}
\frametitle{Programita recursivo (en Python 3) para el Método de
Bisección}


\frametitle{}

\lstinputlisting[firstline=16,lastline=29]{../biseccion.py}

\end{frame}


\begin{frame} 

\frametitle{Programita recursivo (en Python 3) para el Método de
Bisección}


\frametitle{Acá viene el algoritmo de bisección en sí...}

\lstinputlisting[firstline=30,lastline=43]{../biseccion.py}

\end{frame}
\begin{frame}


\frametitle{El otro caso es similar}

\lstinputlisting[firstline=44,lastline=56]{../biseccion.py}

\end{frame}

\part{Un ejemplo}

\frame{\partpage}


\begin{frame}

\frametitle{Un ejemplo de cómo usarlo}

\lstinputlisting[firstline=61,lastline=70]{../biseccion.py}

\end{frame}

% The fragile option is needed with verbatim
% https://tex.stackexchange.com/questions/256666/paragraph-ended-before-verbatim-was-complete-when-trying-to-use-verbatim-in
\begin{frame}[fragile]
\frametitle{Resultado}

{\azul 
\begin{verbatim}
n=  1 :Bisección en el intervalo [ 0 , 2 ]
n=  2 :Bisección en el intervalo [ 1.0 , 2 ]
n=  3 :Bisección en el intervalo [ 1.0 , 1.5 ]
n=  4 :Bisección en el intervalo [ 1.25 , 1.5 ]
n=  5 :Bisección en el intervalo [ 1.375 , 1.5 ]
n=  6 :Bisección en el intervalo [ 1.375 , 1.4375 ]
n=  7 :Bisección en el intervalo [ 1.40625 , 1.4375 ]
n=  8 :Bisección en el intervalo [ 1.40625 , 1.421875 ]
n=  9 :Bisección en el intervalo [ 1.4140625 , 1.421875 ]
n=  10 :Bisección en el intervalo [ 1.4140625 , 1.41796875 ]
n=  11 :Bisección en el intervalo [ 1.4140625 , 1.416015625 ]
n=  12 :Bisección en el intervalo [ 1.4140625 , 1.4150390625 ]
raiz_hallada= 1.4140625
valor exacto= 1.4142135623730951

\end{verbatim}
}
\end{frame}

\begin{frame}

\frametitle{Código fuente}

{\Large

Ustedes pueden descargar el código fuente de esta clase desde

\bigskip

\rojo{
\begin{url}
https://github.com/pdenapo/metodos-numericos
\end{url}
}}

\end{frame}

\end{document}


